% Options for packages loaded elsewhere
\PassOptionsToPackage{unicode}{hyperref}
\PassOptionsToPackage{hyphens}{url}
%
\documentclass[
]{article}
\usepackage{amsmath,amssymb}
\usepackage{iftex}
\ifPDFTeX
  \usepackage[T1]{fontenc}
  \usepackage[utf8]{inputenc}
  \usepackage{textcomp} % provide euro and other symbols
\else % if luatex or xetex
  \usepackage{unicode-math} % this also loads fontspec
  \defaultfontfeatures{Scale=MatchLowercase}
  \defaultfontfeatures[\rmfamily]{Ligatures=TeX,Scale=1}
\fi
\usepackage{lmodern}
\ifPDFTeX\else
  % xetex/luatex font selection
\fi
% Use upquote if available, for straight quotes in verbatim environments
\IfFileExists{upquote.sty}{\usepackage{upquote}}{}
\IfFileExists{microtype.sty}{% use microtype if available
  \usepackage[]{microtype}
  \UseMicrotypeSet[protrusion]{basicmath} % disable protrusion for tt fonts
}{}
\makeatletter
\@ifundefined{KOMAClassName}{% if non-KOMA class
  \IfFileExists{parskip.sty}{%
    \usepackage{parskip}
  }{% else
    \setlength{\parindent}{0pt}
    \setlength{\parskip}{6pt plus 2pt minus 1pt}}
}{% if KOMA class
  \KOMAoptions{parskip=half}}
\makeatother
\usepackage{xcolor}
\usepackage[margin=1in]{geometry}
\usepackage{graphicx}
\makeatletter
\newsavebox\pandoc@box
\newcommand*\pandocbounded[1]{% scales image to fit in text height/width
  \sbox\pandoc@box{#1}%
  \Gscale@div\@tempa{\textheight}{\dimexpr\ht\pandoc@box+\dp\pandoc@box\relax}%
  \Gscale@div\@tempb{\linewidth}{\wd\pandoc@box}%
  \ifdim\@tempb\p@<\@tempa\p@\let\@tempa\@tempb\fi% select the smaller of both
  \ifdim\@tempa\p@<\p@\scalebox{\@tempa}{\usebox\pandoc@box}%
  \else\usebox{\pandoc@box}%
  \fi%
}
% Set default figure placement to htbp
\def\fps@figure{htbp}
\makeatother
\setlength{\emergencystretch}{3em} % prevent overfull lines
\providecommand{\tightlist}{%
  \setlength{\itemsep}{0pt}\setlength{\parskip}{0pt}}
\setcounter{secnumdepth}{-\maxdimen} % remove section numbering
\usepackage{bookmark}
\IfFileExists{xurl.sty}{\usepackage{xurl}}{} % add URL line breaks if available
\urlstyle{same}
\hypersetup{
  pdftitle={Synthèse de l'Analyse - Superstore},
  hidelinks,
  pdfcreator={LaTeX via pandoc}}

\title{Synthèse de l'Analyse - Superstore}
\author{}
\date{\vspace{-2.5em}}

\begin{document}
\maketitle

\section{Synthèse des Résultats}\label{synthuxe8se-des-ruxe9sultats}

L'analyse des données de Superstore a révélé des tendances claires et
des interactions significatives entre les produits, les régions et les
marges bénéficiaires.

\subsection{Performances par Catégorie de
Produits}\label{performances-par-catuxe9gorie-de-produits}

\begin{itemize}
\tightlist
\item
  \textbf{Furniture} : Volume de ventes élevé, mais marge bénéficiaire
  faible (coûts logistiques élevés, remises fréquentes).
\item
  \textbf{Office Supplies} : Ventes stables avec marges modérées,
  surtout pour les produits haut de gamme.
\item
  \textbf{Technology} : Marges élevées, ventes plus rares. Promotions
  efficaces mais réduisent les profits.
\end{itemize}

\subsection{Analyse Régionale}\label{analyse-ruxe9gionale}

\begin{itemize}
\tightlist
\item
  \textbf{West et East} : Très bonnes performances, clientèle corporate,
  commandes volumineuses.
\item
  \textbf{Central et South} : Marges plus faibles à cause d'une
  clientèle plus sensible aux remises.
\end{itemize}

\subsection{Relations
Produits-Régions}\label{relations-produits-ruxe9gions}

\begin{itemize}
\tightlist
\item
  Meubles : populaires dans l'East.
\item
  Technologie : dominante dans le West.
\item
  Fournitures : populaires partout.
\end{itemize}

\subsection{Impact des Remises}\label{impact-des-remises}

\begin{itemize}
\tightlist
\item
  \textbf{\textgreater20\%} : réduisent drastiquement les profits,
  surtout pour les meubles.
\item
  \textbf{10-15\%} : bonnes pour stimuler les ventes sans trop impacter
  les marges.
\end{itemize}

\subsection{Segmentation Clients}\label{segmentation-clients}

\begin{itemize}
\tightlist
\item
  \textbf{Corporate} : 40\% des revenus, marges stables.
\item
  \textbf{Consumer} : sensibles aux promotions mais fidélité faible.
\end{itemize}

\section{Recommandations Métier}\label{recommandations-muxe9tier}

\subsection{Optimisation des Remises}\label{optimisation-des-remises}

\begin{itemize}
\tightlist
\item
  Limiter les remises \textgreater20\% sur les meubles.
\item
  Promouvoir la technologie auprès des clients corporate.
\item
  Introduire des remises progressives.
\end{itemize}

\subsection{Stratégie Régionale}\label{stratuxe9gie-ruxe9gionale}

\begin{itemize}
\tightlist
\item
  Renforcer la distribution dans l'East (meubles) et West (technologie).
\item
  Marketing ciblé dans le South et Central pour transformer les
  consommateurs en clients corporate.
\end{itemize}

\subsection{Gestion des Stocks}\label{gestion-des-stocks}

\begin{itemize}
\tightlist
\item
  Réduire les stocks de meubles en Central.
\item
  Augmenter ceux de fournitures dans le West.
\item
  Logistique juste-à-temps pour la technologie.
\end{itemize}

\subsection{Fidélisation Client}\label{fiduxe9lisation-client}

\begin{itemize}
\tightlist
\item
  Programmes de fidélité pour les consommateurs.
\item
  Services personnalisés pour les clients corporate.
\end{itemize}

\subsection{Analyse Continue}\label{analyse-continue}

\begin{itemize}
\tightlist
\item
  Tableau de bord interactif.
\item
  Audits trimestriels.
\end{itemize}

\section{Conclusion}\label{conclusion}

L'analyse met en lumière des opportunités concrètes pour optimiser la
stratégie commerciale de Superstore. En ajustant les remises, ciblant
mieux les régions et les segments, et en assurant un suivi régulier,
l'entreprise peut améliorer durablement ses marges et son chiffre
d'affaires.

\end{document}
